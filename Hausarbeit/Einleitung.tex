\section{Einleitung} % (fold)
    \label{sec:einleitung}
    Der Mensch hat sich an die Spitze der Nahrungskette gekämpft und diverse Möglichkeiten gefunden, sich gegen jegliche bekannten Lebewesen zu behaupten, die er sehen und anfassen kann. Doch was ist mit denen, die er nicht sehen kann? Die Bakterien und Viren oder kleinste Eukaryoten mit einer parasitären Überlebensstrategie beeinflussen ihre Wirte von innen. Um diesen Gefahren zu begegnen, hat sich in der Evolution ein sehr komplexes und starkes Immunsystem entwickelt, auch ist der Mensch selbst erfinderisch geworden und hat viel in der Medizin geforscht, um mit den Mikroorganismen auszukommen. Dennoch hat sich jüngst ein hoch infektiöses Virus (COVID-19) sehr schnell weltweit verbreitet und führte bei Millionen von Menschen zum Tode\ \cite[vgl.][]{corona}, wo sich nun die Frage stellt, wann sich der nächste gefährliche Mikroorganismus entwickelt und verbreitet, mit womöglich noch mehr Todesfällen.

    Man stelle sich eine Zombie-Apokalypse vor, ausgelöst durch einen Parasit, der seinem Wirt jegliche Persönlichkeit nimmt und anschließend als wandelndes Werkzeug nutzt, um weitere Lebewesen zu infizieren. Dieser Parasit wird durch Körperkontakt übertragen, da eine Übertragung über die Luft wohl kaum aufzuhalten wäre, wie oben genannter Virus vermuten lässt. Der erste seiner Art taucht in einer kleinen Stadt auf, und die Pandemie nimmt ihren Lauf. Dieses Szenario soll nun simuliert werden, um die Möglichkeiten unserer Spezies zu ermitteln, also ob der Parasit ausgerottet werden könnte, eine Koexistenz möglich oder die Menschheit dem Tode geweiht wäre.

    Nicht nur in der unterhaltenden Fiktion\ \cite{zombie_fiction}, sondern auch in der Wissenschaft gibt es schon diverse Begegnungen mit dieser Problematik. In dem Buch ``Mathematical Modelling of Zombies'' wird mittels mathematischen Modellen, Differentialgleichungen und Statistik eine Zombie-Apokalypse simuliert, deren Verlauf und unsere Möglichkeiten berechnet, mit dem Ergebnis, dass ein Überleben möglich wäre\ \cite{zombie_smith}. Auf Basis dessen haben sich auch andere Autoren mit diesem Thema beschäftigt, mit Augenmerk auf die Veränderungen der Stabilität durch Bifurkation in einem solchen Modell\ \cite[vgl.][]{zombie_science2}. Sie kommen auch zu dem Ergebnis, dass es möglich sei, diesen Ausbruch zu überleben.

    In dieser Arbeit werden auch die menschlichen Möglichkeiten analysiert, allerdings mit dem Unterschied, dass ein sehr vereinfachtes Modell der Apokalypse verwendet wird. Zwar beeinflussen in der Realität sehr viele Faktoren den Ausbruch und die Verbreitung des Parasiten, die in genannter Literatur auch zum Teil beachtet werden, dennoch können zu viele Parameter den Blick auf das Offensichtliche oder unvorstellbar aber Mögliche verschleiern, weshalb diese in Folgendem auf ein Minimum heruntergebrochen werden.
% section einleitung (end)