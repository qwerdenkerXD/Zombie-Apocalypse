\section{Material und Methoden} % (fold)
    \label{sec:material_und_methoden}
    Um die Simulation durchzuführen, wurde als Modell eine Stadt mit circa 20.597 Einwohnern verwendet. Die Bevölkerung unterteilt sich darin in vier sogenannte Spezies:
    \begin{enumerate}[1.]
        \item \textbf{Menschen}: Ganz normale Einwohner ohne besondere Fähigkeiten. Im Jäger-und-Sammler-Modell wären sie die Sammler und meiden daher den Kampf gegen Zombies.\label{species}
        \item \textbf{Helden}: Taffe Einwohner mit besonders ausgeprägtem Überlebensinstinkt. Sie jagen aktiv die Zombies, um sich ihr friedliches Leben zurückzuerobern.
        \item \textbf{Zombies}: Verseuchte Einwohner. Der Parasit hat diese ehemalig Lebenden abgetötet und steuert nun den toten Körper, um weitere Einwohner zu infizieren.
        \item \textbf{Tote}: Endgültig tote Einwohner, die vom Parasiten auch nicht mehr als Wirt benutzt werden können.
    \end{enumerate}
    Mit diesem Modell wurde nun ein Python-Skript entwickelt (\texttt{Supplement/simulation.py}), welches die Entwicklung der Bevölkerung, also die Anzahlen der Individuen der Spezies, über einen selbst gegebenen Zeitraum simuliert und als Plot visualisiert. Um dies möglichst realistisch und akkurat abzubilden, wurde die Simulation schrittweise erarbeitet:
    \begin{enumerate}[1.]
        \item \textbf{Normalbedingungen}\ \ Zuerst wurde die Population bei gewöhnlichen Bedingungen betrachtet, also das Leben ohne Pandemie. Repräsentativ kam hier die Stadt Zülpich zur Verwendung. Diese hat Ende 2021 etwa 20.597 Einwohner gehabt. In dem Jahr sind 191 lebende Kinder geboren worden, 287 Menschen gestorben, 1245 zugezogen und 991 fortgezogen \cite{zulpich}. Diese heruntergeladenen Daten sind in \texttt{Supplement/Zulpich} einsehbar und wurden für die normale Bevölkerungsentwicklung einbezogen. Dabei beeinflussen die Geburten und Todesfälle die lebende Bevölkerung, also die Menschen und Helden, prozentual, da sich bei wachsender oder sinkender Bevölkerungszahl diese allgemeinen Werte mitverändern, während die Fort- und Zuzüge mehr von der Attraktivität der Stadt abhängen, deren Einfluss hier vernachlässigt wurde.
        \item \textbf{Keine Helden}\ \ Die Normalbedingungen korrekt simuliert, wurde nun der Einfluss der Menschen auf die Pandemie justiert. Dementsprechend erhielten die Helden eine unveränderliche Anzahl von 0, sodass anhand von Testsimulationen aus den resultierenden Plots die \hyperref[species]{oben} beschriebene Interaktion zwischen Menschen und Zombies korrekt implementiert werden konnte... folgende Funktionen und so...
    \end{enumerate}

% section material_und_methoden (end)
