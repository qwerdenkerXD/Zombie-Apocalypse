\phantomsection  % important for correct reference in toc
\label{sec:abstract}
\addcontentsline{toc}{section}{Abstract}
    \begin{center}
      \textbf{Zusammenfassung}
    \end{center}
    Nachdem sich der Mensch über Generationen eine Vormachtstellung in der Nahrungskette erkämpft hat, können ihn wohl nur noch zwei Spezies ernsthafte Schwierigkeiten bereiten. Einerseits er selbst, andererseits kleinste parasitäre Organismen, die ein schweres Ziel darstellen. Um für eine durch diese Parasiten ausgelöste Pandemie Präventivmaßnahmen zu erörtern, wird in folgender Arbeit für eine solche eine Simulation modelliert, anhand derer die Möglichkeiten der Menschen in so einem Ernstfall ermittelt werden sollen. Exemplarisch wird dieses Modell auf einer Zombie-Apokalypse gründen, also einem Szenario, in dem der Parasit die Menschen gegen sich selbst ausspielt, mit dem Ergebnis, das tatsächlich die Möglichkeit besteht, so einen Ausbruch zu überleben.
\vspace{4em}
    \begin{otherlanguage}{english}
        \begin{center}
          \textbf{Abstract}
        \end{center}
        After humanity has fought for a dominant position in the food chain over generations, only two species may pose serious challenges. On one hand, humans themselves, and on the other, minute parasitic organisms that present a difficult target. To discuss preventive measures for a pandemic caused by these parasites, this thesis will model a simulation to determine human capabilities in such a critical situation. This model will be based on a zombie apocalypse scenario, where the parasite pits humans against each other, with the result that there is indeed a possibility of surviving such an outbreak.
    \end{otherlanguage}