\section{Diskussion} % (fold)
    \label{sec:diskussion}
    Betrachtet man \autoref{fig:simulation}, deutet das Ergebnis darauf hin, dass das Überleben einer Zombieapokalypse möglich wäre. Selbst dieses einfache Modell stimmt mit den Resultaten der aktuellen Wissenschaft überein. Dennoch sind hier einige Dinge zu beachten. Durch die wahrscheinlichkeitsbasierte Simulation bei so wenigen Parametern, können geringste Veränderungen der Startkonfiguration ein komplett anderes Ergebnis liefern, sodass selbst im Modell ohne Helden die Menschen am Anfang glücklicherweise die ersten Kämpfe gewinnen und es somit gar nicht erst zu einem richtigen Ausbruch kommt, oder die Gesamtsimulation bei etwas anderen Individuenzahlen in der Katastrophe endet. Auch ist dies hier ein Modellparasit. Es ist unklar, ob die Gewinnchancen letztendlich wirklich so verteilt sind oder die Zombies vieleicht stärker oder schwächer sind.

    Trotzdem kann man aus den Simulationen einige Thesen aufstellen, die ein positives Resultat begünstigen könnten:
    \begin{enumerate}[1.]
        \item \textbf{Helden sind hilfreich}:
            Es empfiehlt sich, wenn ein Teil der Bevölkerung etwas davon versteht, sich im Ernstfall angemessen verteidigen zu können. \autoref{fig:simulation} wäre in der Katastrophe geendet, wenn es nicht Helden gegeben hätte, die gegen die wachsende Zahl an Zombies ankämpfte.
        \item \textbf{Kampferfahrungen zu teilen ist wichtig}:
            Vergleicht man \autoref{fig:no_humans} und \autoref{fig:simulation}, stellt sich die Frage, wie es sein kann, dass etwa 20000 Helden gegen 2000 Zombies verlieren, während sie in \autoref{fig:simulation} am Ende sogar in Unterzahl waren. Der Grund ist die Kampferfahrung. Die Gewinnwahrscheinlichkeit der Helden steigt mit der Zeit, wie \hyperref[steps:no_humans]{oben} beschrieben, da währenddessen die Schwächen der Zombies ausgemacht werden können. Diese Lernphase dauert 30 Tage, sodass die Helden in \autoref{fig:simulation} am Ende ihre nahezu vollständige Stärke erreicht haben.

            In \autoref{fig:no_humans} haben sie diese noch nicht und verlieren dementsprechend. Simuliert wird hier also eher der Fall, dass die Stadt von den Zombies übernommen wurde und nun eine Armee hineingeschickt wird, um die übrigen zu erledigen, ohne die ganze Stadt zerstören zu müssen. Daher ist es also notwendig, dass die vorigen Einwohner ihre Erfahrungen möglichst schnell verbreiten, damit diese `Kennlernphase' der Helden der Verstärkung übersprungen werden kann
    \end{enumerate}
% section diskussion (end)