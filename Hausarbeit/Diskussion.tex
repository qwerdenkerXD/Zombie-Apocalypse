\section{Diskussion} % (fold)
    \label{sec:diskussion}
    Betrachtet man \autoref{fig:simulation}, deutet das Ergebnis darauf hin, dass das Überleben einer Zombieapokalypse möglich wäre. Selbst dieses einfache Modell stimmt mit den Resultaten der aktuellen Wissenschaft überein. Dennoch sind hier einige Dinge zu beachten. Durch die wahrscheinlichkeitsbasierte Simulation bei so wenigen Parametern, können geringste Veränderungen der Startkonfiguration ein komplett anderes Ergebnis liefern, sodass selbst im Modell ohne Helden die Menschen am Anfang glücklicherweise die ersten Kämpfe gewinnen und es somit gar nicht erst zu einem richtigen Ausbruch kommt, oder die Gesamtsimulation bei etwas anderen Individuenzahlen in der Katastrophe endet. Auch ist dies hier ein Modellparasit. Es ist unklar, ob die Gewinnchancen letztendlich wirklich so verteilt sind oder die Zombies vieleicht stärker oder schwächer sind.

    Trotzdem kann man aus den Simulationen einige Thesen aufstellen, die ein positives Resultat begünstigen könnten:
    \begin{enumerate}[1.]
        \item \textbf{Helden sind hilfreich}:
            Es empfiehlt sich, wenn ein Teil der Bevölkerung etwas davon versteht, sich im Ernstfall angemessen verteidigen zu können. \autoref{fig:simulation} wäre in der Katastrophe geendet, wenn es nicht Helden gegeben hätte, die gegen die wachsende Zahl an Zombies ankämpfte.

        \item \textbf{Helden sollten hilfsbereit sein}:
            Zwar sind Helden an sich schon von Vorteil, dennoch sollten sie auch ein wenig selbstlos sein, so wie in diesem Modell, und ihr Leben riskieren, um Menschen vor Zombies zu retten und damit möglicherweise neue Helden zu rekrutieren. Aus \autoref{fig:simulation} kann man ableiten, dass ohne diesen dadurch resultierenden Zuwachs wohl die Zombies gewonnen hätten.

        \item \textbf{Kampferfahrungen zu teilen ist wichtig}:
            Vergleicht man \autoref{fig:no_humans} und \autoref{fig:simulation}, stellt sich die Frage, wie es sein kann, dass etwa 20000 Helden gegen 2000 Zombies verlieren, während sie in \autoref{fig:simulation} am Ende sogar in Unterzahl waren. Der Grund ist die Kampferfahrung. Die Gewinnwahrscheinlichkeit der Helden steigt mit der Zeit, wie \hyperref[steps:no_humans]{oben} beschrieben, da währenddessen die Schwächen der Zombies ausgemacht werden können. Diese Lernphase dauert 30 Tage, sodass die Helden in \autoref{fig:simulation} am Ende ihre nahezu vollständige Stärke erreicht haben.

            In \autoref{fig:no_humans} haben sie diese noch nicht und verlieren dementsprechend. Simuliert wird hier also eher der Fall, dass die Stadt von den Zombies übernommen wurde und nun eine Armee hineingeschickt wird, um die übrigen zu erledigen, ohne die ganze Stadt zerstören zu müssen. Daher ist es also notwendig, dass die vorigen Einwohner ihre Erfahrungen möglichst schnell verbreiten, damit diese `Kennlernphase' der Helden der Verstärkung übersprungen werden kann
    \end{enumerate}
    Auch wenn das Maßnahmen sind, die möglicherweise zum Sieg verhelfen, gibt es doch trotzdem Faktoren, die wohl zu einer unvermeidbaren Katastrophe führen würden. Einer wurde schon in der \hyperref[sec:einleitung]{Einleitung} beim Beschreiben des Szenarios angesprochen, nämlich ein Parasit, der über die Luft übertragen wird. In den Simulationen haben Verwandlungen immer nur bei direktem Kontakt mit den Zombies stattgefunden, bei Luftübertragung müsste es nicht einen Kampf geben und die gesamte Bevölkerung könnte zu Zombies geworden sein, von der überstädtischen Ausbreitung abgesehen, worauf gleich näher eingegangen wird, da diese hier vernachlässigt wurde.

    Ein weiterer Faktor ist die Startzahl an Infizierten. Angenommen, die Pandemie beginnt in einem Kindergarten. Rasant würden alle Kinder infiziert sein, folglich auch die Betreuer, die wahrscheinlich nicht bereit wären, gegen sie zu kämpfen. Zuletzt kämen noch die Eltern, die ihre Kinder abholen wollen. In diesem Fall beginnt alles nicht mit einem einzigen Infizierten, sondern gleich mit einer ganzen Schar, die folglich um ein Vielfaches schwerer einzudämmen wäre.

    Zuletzt spielt es außerdem eine Rolle, wie die Infizierten erkannt werden können. In dem erstellten Modell werden sie durch ihre agressive Art erkannt, andere fressen zu wollen und sie dadurch in den Kampf verwickeln. Aber wesentlich schwerer wäre es für die Lebenden, wenn die Zombies unerkannt unter ihnen weilen, bis sich eine günstige Gelegenheit bietet, sie zu verwandeln. Sie wären somit ein weniger offensichtliches Ziel, gegen das vorgegangen werden könnte. Wegen der möglichen Panik und dem Misstrauen zu anderen, müssten die Funktion \texttt{human\_kills\_human} und die Gewinnwahrscheinlichkeiten definitiv erweitert werden.

    Was die überstädtische Ausbreitung betrifft, wurde diese in diesem Modell nicht mit eingeschlossen. Geht man davon aus, dass ein paar Zombies aus der Stadt in eine andere umsiedeln, könnte man die Simulation einer Stadt auf alle übertragen, sodass es am Ende eine Menge an Städten mit Lebenden und eine Menge mit Zombies gäbe. In diesem Fall erschließe sich auch ein neuer möglicher Ausgang, nämlich einer Koexistenz beider Spezies. Die Lebenden hätten mit ihrer Stadt eine Art Basis, die sie vor den Feinden von außen beschützen könnten, da der Feind von innen schließlich besiegt wurde. Problematisch beim Einbezug der Ausbreitung sind allerdings die geographischen und geologischen Verhältnisse, die überall anders sind. Diese in diesem Modell realistisch umzusetzen, ohne dabei die Einfachheit zu verletzen, wäre wohl sehr schwierig und sollte dann wohl eher in einer separaten Simulation erforscht werden. Sehr empfehlenswert wäre es vermutlich, bei erstem bekannten Kontakt mit einem Zombie, alle Inseln von der Welt abzuschotten, um so im schlimmsten Fall wenigstens dort ein Überleben zu gewährleisten.

    Auch zu einer Koexistenz würde ein Heilmittel führen. Während die Lebenden nämlich durch die Kämpfe nur Verluste einstreichen, sind die Zombies durch die Verwandlung in der Lage, ihre Niederlagen zu kompensieren. Das ist ein ungleicher Zustand, der durch ein Heilmittel ausgeglichen würde, da in dem Fall auch die Lebenden diese Chance hätten. Dies setzt aber voraus, dass das Heilmittel schnell verfügbar ist und effizient verteilt werden kann, sofern es denn eines gibt, dass in sehr kurzer Zeit entwickelt werden könnte, um eine Rolle zu spielen. Eher denkbar ist, dass diese Art der Koexistenz die vorher Beschriebene ablöst, da dann nämlich genug Zeit vorhanden wäre, um es zu entdecken.

    All diese Folgerungen und Vermutungen zusammengenommen, bleibt es weiterhin bei dem Ergebnis, dass ein Überleben möglich wäre. Ob durch Koexistenz oder Besiegen der Zombies, gibt es Maßnahmen, die ergriffen werden können, um präventiv für die besten Chancen zu sorgen, auch wenn es letztenendes vom Zufall abhängt, wie der Parasit funktioniert, um sein eigenes Überleben zu garantieren.
% section diskussion (end)